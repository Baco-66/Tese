%!TEX root = ../template.tex
%%%%%%%%%%%%%%%%%%%%%%%%%%%%%%%%%%%%%%%%%%%%%%%%%%%%%%%%%%%%%%%%%%%%
%% abstract-pt.tex
%% NOVA thesis document file
%%
%% Abstract in Portuguese
%%%%%%%%%%%%%%%%%%%%%%%%%%%%%%%%%%%%%%%%%%%%%%%%%%%%%%%%%%%%%%%%%%%%

\typeout{NT FILE abstract-pt.tex}%


Nas últimas três décadas, a procura de integração da conectividade à Internet no domínio das redes veiculares, normalmente designadas por \gls{vanet}, tem vindo a aumentar. \gls{sdn} representa um novo paradigma de rede que promete numerosos benefícios, e cuja concretização pode potencialmente melhorar as \glspl{vanet}. Esta linha de pensamento levou ao aparecimento da \gls{sdvn}, sendo um dos principais obstáculos à sua adoção a falta de plataformas experimentais baseadas em hardware real. Este problema representa um desafio significativo para a investigação em \gls{sdvn}, uma vez que é altamente complexo e exige uma compreensão abrangente dos domínios \gls{sdn} e \gls{vanet}. Além disso, existe uma escassez de investigação que examine a integração da \gls{p4} e da segunda geração de \gls{sdn} com as \glspl{vanet}. Dada a natureza inovadora deste desafio e a complexidade excecional destes dois domínios, a sua resolução constitui um objetivo altamente interessante e ambicioso.

Este documento tem por objetivo colmatar a lacuna existente neste domínio de estudo, apresentando o desenvolvimento e a implementação de um \gls{obu}/\gls{its} vehicle subsystem que cumpra as especificações de \gls{sdn}. O protótipo proposto baseia-se em hardware acessível e facilmente disponível e utiliza software de código aberto sempre que possível. Além disso, este documento apresenta uma nova perspetiva sobre o futuro do domínio \gls{sdvn}, com base numa análise global abrangente. Estas conclusões apresentam uma potencial trajetória para a evolução da tecnologia \gls{sdvn}. Este artigo apresenta o desenvolvimento de um protótipo constituído por um PC Engines apu3d4 que opera o switch \gls{p4} \gls{bmv2}, comunicando a frequências de 5,9 GHz através da norma 802.11p. Esta solução estabelece as bases necessárias para o desenvolvimento de um ambiente \gls{sdvn} completo no qual podem ser efectuados testes.


% Palavras-chave do resumo em Português
% \begin{keywords}
% Palavra-chave 1, Palavra-chave 2, Palavra-chave 3, Palavra-chave 4
% \end{keywords}
\keywords{
	Redes vehiculares \and
	Redes definidas por software \and
	\glsxtrlong{its} \and
	Redes vehiculares definidas por software \and
	\glsxtrlong{etsi} \and
	\glsxtrlong{obu} \and
	\glsxtrshort{its} vehicle subsystem \and
	802.11p \and
	\glsxtrlong{ovs} \and
	\glsxtrlong{bmv2} \and
	\glsxtrlong{p4} 
}
% to add an extra black line



