%!TEX root = ../template.tex
%%%%%%%%%%%%%%%%%%%%%%%%%%%%%%%%%%%%%%%%%%%%%%%%%%%%%%%%%%%%%%%%%%%%
%% abstract-en.tex
%% NOVA thesis document file
%%
%% Abstract in English([^%]*)
%%%%%%%%%%%%%%%%%%%%%%%%%%%%%%%%%%%%%%%%%%%%%%%%%%%%%%%%%%%%%%%%%%%%

\typeout{NT FILE abstract-en.tex}%



Concerning its contents, the abstracts should not exceed one page and may answer the following questions (it is essential to adapt to the usual practices of your scientific area):

\begin{enumerate}
  \item What is the problem?
  \item Why is this problem interesting/challenging?
  \item What is the proposed approach/solution/contribution?
  \item What results (implications/consequences) from the solution?
\end{enumerate}

% What's the problem?
%\noindent In recent years, there has been a growing demand to bring internet connectivity to moving vehicles. However, traditional internet protocols and systems cannot handle the unique challenges posed by this environment. Software Defined Networking (SDN) has emerged as a potential solution, with one of the main barriers to adoption being a lack of real-world testing. To address this gap, this paper presents the development and implementation of an On-Board Unit (OBU) device fully compatible with SDN. The OBU device is designed to address the challenges of providing internet connectivity to moving vehicles and to be compatible with any Software Defined Vehicular Networking (SDVN) scenario required.

% Why is it interesting?
%Therefore, this paper's objective is to develop an OBU device that is SDN compatible.  
%There is a huge lack of real hardware to test SDVNs.

% What's the solution?
%The solution is to build a device using opensouce software and cheap hardware.

% What follows from the solution?




% Palavras-chave do resumo em Inglês
% \begin{keywords}
% Keyword 1, Keyword 2, Keyword 3, Keyword 4, Keyword 5, Keyword 6, Keyword 7, Keyword 8, Keyword 9
% \end{keywords}
\keywords{
    Vehicular networks \and
    Software defined networks
}



