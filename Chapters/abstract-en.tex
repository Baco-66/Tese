%!TEX root = ../template.tex
%%%%%%%%%%%%%%%%%%%%%%%%%%%%%%%%%%%%%%%%%%%%%%%%%%%%%%%%%%%%%%%%%%%%
%% abstract-en.tex
%% NOVA thesis document file
%%
%% Abstract in English([^%]*)
%%%%%%%%%%%%%%%%%%%%%%%%%%%%%%%%%%%%%%%%%%%%%%%%%%%%%%%%%%%%%%%%%%%%

\typeout{NT FILE abstract-en.tex}%

% What's the problem?

% Why is this problem interesting/challenging?
% Why is it interesting?

% What's the solution?
% What is the proposed approach/solution/contribution?

% What follows from the solution?
% What results (implications/consequences) from the solution?

Over the past three decades, there has been a surge in demand for the integration of internet connectivity within the domain of vehicular networks, commonly referred to as \glsxtrshortpl{vanet}. 
\glsxtrshort{sdn} represents a novel networking paradigm that promises numerous benefits, the realization of which could potentially enhance \glsxtrshortpl{vanet}. 
A promising approach that has emerged involves the application of \glsxtrshort{sdn} to vehicular networks, a concept referred to as \glsxtrshort{sdvn}. 
However, one of the main obstacles to its use is the lack of experimental platforms using real hardware.
This problem represents a significant challenge for \glsxtrshort{sdvn} research, as it is highly intricate and necessitates a comprehensive grasp of both \glsxtrshort{sdn} and \glsxtrshort{vanet} domains. 
Furthermore, the integration of \glsxtrshort{p4} and the second generation of \glsxtrshort{sdn} with \glsxtrshortpl{vanet} remains an under-explored field, thereby creating new research opportunities. 
Given the innovative nature of this challenge and the complexity of these two domains, addressing it is a highly interesting and ambitious goal.

This dissertation seeks to make a meaningful contribution to the field by presenting the development and implementation of a vehicle \glsxtrshort{obu}/Vehicle \glsxtrshort{its} sub-system that complies with the specifications of \glsxtrshort{sdn}.
The proposed prototype is based on affordable and readily available hardware and utilizes open-source software wherever feasible. 
This system features an embedded PC Engines apu3d4 system, with a wireless communication card capable of operating in the 5.9 GHz band according to the \glsxtrshort{ieee} 802.11p standard for \glsxtrshort{v2x} communications. It also  features a \glsxtrshort{bmv2} switch, which enables support for \glsxtrshort{p4}. 
The system successfully underwent testing in a series of previously defined experimental scenarios, thereby meeting the requirements established for its use as a test platform in real-world environments. 
Additionally, this paper provides a new perspective on the future of the \glsxtrshort{sdvn} field, based on a comprehensive global analysis. These findings present a potential trajectory for the evolution of \glsxtrshort{sdvn} technology. This paper presents the development of a prototype comprising a PC Engines apu3d4 that operates the \glsxtrshort{bmv2} \glsxtrshort{p4} switch, communicating at 5.9GHz frequencies via the 802.11p standard. 


This solution establishes the necessary framework for the assembly of a comprehensive \glsxtrshort{sdvn} environment. 




% Palavras-chave do resumo em Inglês
% \begin{keywords}
% Keyword 1, Keyword 2, Keyword 3, Keyword 4, Keyword 5, Keyword 6, Keyword 7, Keyword 8, Keyword 9
% \end{keywords}
\keywords{
  \glsxtrlong{vanet} \and
  \glsxtrlong{sdn} \and
  \glsxtrlong{its} \and
  \glsxtrlong{sdvn} \and
  \glsxtrlong{obu} \and
  Vehicle \glsxtrshort{its} subsystem \and
  \glsxtrshort{ieee} 802.11p \and
  \glsxtrlong{p4} 
}


