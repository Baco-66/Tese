%!TEX root = ../template.tex
%%%%%%%%%%%%%%%%%%%%%%%%%%%%%%%%%%%%%%%%%%%%%%%%%%%%%%%%%%%%%%%%%%%%
%% abstract-en.tex
%% NOVA thesis document file
%%
%% Abstract in English([^%]*)
%%%%%%%%%%%%%%%%%%%%%%%%%%%%%%%%%%%%%%%%%%%%%%%%%%%%%%%%%%%%%%%%%%%%

\typeout{NT FILE abstract-en.tex}%

% What's the problem?

% Why is this problem interesting/challenging?
% Why is it interesting?

% What's the solution?
% What is the proposed approach/solution/contribution?

% What follows from the solution?
% What results (implications/consequences) from the solution?

Over the past three decades, there has been a surge in demand for the integration of internet connectivity within the domain of vehicular networks, commonly referred to as \glspl{vanet}. \gls{sdn} represents a novel networking paradigm that promises numerous benefits, the realization of which could potentially enhance \glspl{vanet}. This line of thinking has led to the emergence of \gls{sdvn}, with one of the primary obstacles to its adoption being the lack of experimental platforms utilizing real hardware. This problem represents a significant challenge for \gls{sdvn} research, as it is highly intricate and necessitates a comprehensive grasp of both \gls{sdn} and \gls{vanet} domains. Moreover, there is a dearth of research examining the integration of \gls{p4} and the second generation of \gls{sdn} with \glspl{vanet}. Given the novel nature of this endeavor and the exceptional complexity of these domains, solving this challenge presents a highly intriguing, yet difficult, objective.

This paper aims to address the existing gap in this field of study by presenting the development and implementation of an \gls{obu}/\gls{its} vehicle subsystem that complies fully with the specifications of \gls{sdn}. The proposed prototype is based on affordable and readily available hardware and utilizes open-source software wherever feasible. Additionally, this paper provides a new perspective on the future of the \gls{sdvn} field, based on a comprehensive global analysis. These findings present a potential trajectory for the evolution of \gls{sdvn} technology. This paper presents the development of a prototype comprising a PC Engines apu3d4 that operates the \gls{bmv2} \gls{p4} switch, communicating at 5.9GHz frequencies via the 802.11p standard. This solution establishes the necessary framework for the assembly of a comprehensive \gls{sdvn} environment in which tests may be conducted. 




% Palavras-chave do resumo em Inglês
% \begin{keywords}
% Keyword 1, Keyword 2, Keyword 3, Keyword 4, Keyword 5, Keyword 6, Keyword 7, Keyword 8, Keyword 9
% \end{keywords}
\keywords{
  \glsxtrlong{vanet} \and
  \glsxtrlong{sdn} \and
  \glsxtrlong{its} \and
  \gls{sdvn} \and
  \glsxtrlong{etsi} \and
  \glsxtrlong{obu} \and
  \glsxtrshort{its} vehicle subsystem \and
  802.11p \and
  \glsxtrlong{ovs} \and
  \glsxtrlong{bmv2} \and
  \glsxtrlong{p4} 
}


