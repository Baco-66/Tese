%!TEX root = ../template.tex
%%%%%%%%%%%%%%%%%%%%%%%%%%%%%%%%%%%%%%%%%%%%%%%%%%%%%%%%%%%%%%%%%%%%
%% conclusion.tex
%% NOVA thesis document file
%%
%% Chapter with lots of dummy text
%%%%%%%%%%%%%%%%%%%%%%%%%%%%%%%%%%%%%%%%%%%%%%%%%%%%%%%%%%%%%%%%%%%%
\chapter{Conclusions and Future work}
\label{cha:conclusion}

This chapter presents a synopsis of the work conducted, accompanied by a set of conclusions and a proposal for future research directions.

% Conclusion
\section{Conclusions}
In this dissertation, the main objective was to develop an \gls{obu} in accordance with the principles of \gls{sdn}, thereby aligning this document with the domain of \gls{sdvn}. Throughout the course of this project, a critical examination of the actual implications of this goal was undertaken. The findings of the research revealed that the objective of making an \gls{obu} \gls{sdn} compatible is, in essence, equivalent to developing an \gls{sdn} Vehicle \gls{its} sub-system. Moreover, the insights gained from the research enabled the presentation of a feasible and realistic trajectory for \gls{sdvn}. This vision is grounded in the recognition that, for \gls{sdvn} to be deployed in the future, interoperability with traditional devices is of paramount importance. In essence, the success of \gls{sdvn} will be contingent upon its capacity to integrate with existing infrastructure.
In the practical implementation stage, a series of tests were conducted on the most relevant software for data plane emulation in the field of \gls{sdn} in order to ascertain the most appropriate software for vehicular scenarios. In accordance with the requirements of this dissertation, the queries pertaining to the control plane were set aside in order to prioritize the data plane, thereby facilitating the recreation of the \gls{obu}. 
Prototype devices were configured and the \gls{bmv2} and \gls{ovs} software were installed on them. In the absence of any notable requirements, \gls{onos} was chosen as the control plane technology. Subsequently, the ath9k patch was implemented with the objective of evaluating the compatibility of communication utilizing the intended \gls{ieee} 802.11p protocol. The results of the tests revealed that \gls{ovs} was incompatible with wireless interfaces, whereas \gls{bmv2} demonstrated no such incompatibility.
The tests conducted on \gls{ovs} and \gls{bmv2} served to reinforce the hypothesis that \gls{p4} is an adequate data plane technology. The \gls{p4} language, in conjunction with the controller, was employed to modify the subsequent layers, thereby establishing communication in \gls{ieee} 802.11p between two devices. It was thus demonstrated that \gls{bmv2} represents the optimal software for utilization in \gls{sdvn}. 
The present dissertation has been completed in accordance with the stipulated criteria and objectives. All of the proposed goals have been achieved, thereby providing a crucial step in establishing a foundation for the implementation of a complete \gls{sdvn} testing scenario based on real hardware. Ultimately, the research, work, and subsequent findings presented in this document provide sufficient evidence to demonstrate that the integration of \gls{sdn} principles in the field of VANETs is a viable and achievable proposition.

% Future work
\section{Future work}
The natural progression of this work is to adapt the remaining \gls{etsi} sub-systems to be \gls{sdn}-compatible. This process should begin with the Central and Roadside sub-systems, as they represent the most critical components in this context. 
Upon completion of this phase, the foundation will be established for a comprehensive, standards-compliant \gls{sdn} system implementation. This second phase would entail the development of the necessary controller and data plane logic to achieve intercommunication between \gls{sdn} and non-\gls{sdn} devices in conformance with established standards.
The most desirable long-term objective would be the recreation of a \gls{sdn} device that is an exact functional replica of a traditional device. Once this has been achieved, the technology can be implemented and evaluated in conjunction with actual road networks, in a controlled but relevant environment.
However, the complete system should not be regarded as a final stage of research but rather as the foundation for a new line of inquiry. The potential of \gls{p4} can be harnessed to facilitate the testing of a multitude of aspects pertaining to the \gls{etsi} protocols. The first and most crucial task, which is not directly related to the \gls{etsi} protocols, is to identify the optimal controller placement. This, along with related inquiries such as the evaluation of election algorithms for the controller in clusters of vehicles, would represent a vital priority for ensuring the success and advancement of this technology.
Finally, this dissertation also lays the groundwork for the redesign and upgrade of the device used on the \gls{obu}. The patch currently in use to achieve \gls{ieee} 802.11p functionality may not be a long-term solution. Therefore, it would be prudent to explore a more permanent solution, such as \gls{sdr}.


\vfill

This document was created with the (pdf/Xe/Lua)\LaTeX\ processor and the \href{https://github.com/joaomlourenco/novathesis}{NOVAthesis} template (v\novathesisversion)~\cite{novathesis-manual}.

