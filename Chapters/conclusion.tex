%!TEX root = ../template.tex
%%%%%%%%%%%%%%%%%%%%%%%%%%%%%%%%%%%%%%%%%%%%%%%%%%%%%%%%%%%%%%%%%%%%
%% conclusion.tex
%% NOVA thesis document file
%%
%% Chapter with lots of dummy text
%%%%%%%%%%%%%%%%%%%%%%%%%%%%%%%%%%%%%%%%%%%%%%%%%%%%%%%%%%%%%%%%%%%%
\chapter{Conclusion and Future work}
\label{cha:conclusion}

% Conclusion
In this thesis, it was this author's intention to develop an OBU in accordance with the principles of \gls{sdn}, thereby aligning this thesis with the domain of SDVN. Throughout the course of this project, a critical examination of the actual implications of this goal was undertaken. The findings of the research revealed that the objective of making an OBU \gls{sdn} compatible is, in essence, equivalent to developing an \gls{sdn} ITS vehicle subsystem. Moreover, the insights gained from the research enabled the presentation of a feasible and realistic trajectory for SDVN. This vision is grounded in the recognition that, for SDVN to be deployed in the future, interoperability with traditional devices is of paramount importance. In essence, the success of SDVN will be contingent upon its capacity to integrate with existing infrastructure.
In the practical implementation stage, a series of tests were conducted on the most relevant software for data plane emulation in the field of \gls{sdn} in order to ascertain the most appropriate software for vehicular scenarios. In accordance with the requirements of this thesis, the queries pertaining to the control plane were set aside in order to prioritize the data plane, thereby facilitating the recreation of the OBU. 
Prototype devices were configured and the BMv2 and OvS software were installed on them. In the absence of any notable requirements, ONOS was chosen as the control plane technology. Subsequently, the ath9k patch was implemented with the objective of evaluating the compatibility of communication utilizing the intended 802.11p protocol. The results of the tests revealed that OvS was incompatible with wireless interfaces, whereas BMv2 demonstrated no such incompatibility.
The tests conducted on OVS and BMv2 served to reinforce the hypothesis that P4 is the most optimal data plane technology. The P4 language, in conjunction with the controller, was employed to modify the subsequent layers, thereby establishing communication in 802.11p between two devices. It was thus demonstrated that BMv2 represents the optimal software for utilization in SDVN. 
The present thesis has been completed in accordance with the stipulated criteria and objectives. All of the proposed goals have been achieved, thereby providing a crucial step in establishing a foundation for the implementation of a complete SDVN testing scenario based on real hardware. Ultimately, the research, work, and subsequent findings presented in this thesis provide sufficient evidence to demonstrate that the integration of \gls{sdn} principles in the field of VANETs is a viable and achievable proposition.

% Future work
The natural progression of this work is to adapt the remaining ETSI subsystems to be \gls{sdn}-compatible. This process should begin with the Central and Roadside subsystems, as they represent the most critical components in this context. 
Upon completion of this phase, the foundation will be established for a comprehensive, standards-compliant \gls{sdn} system implementation. This second phase would entail the development of the requisite controller and data plane logic to achieve intercommunication between \gls{sdn} and non-\gls{sdn} devices in conformance with established standards.
The most desirable long-term objective would be the recreation of a \gls{sdn} system that is an exact functional replica of a traditional device. Once this has been achieved, the technology can be implemented and evaluated in conjunction with actual road networks.
It would, however, be a mistake to view the complete system as the final stage of research. Rather, it should be regarded as the foundation for a new line of enquiry. The potential of P4 can be harnessed to facilitate the testing of a multitude of aspects pertaining to the ETSI protocols. The first and most crucial task, which is not directly related to the ETSI protocols, is to identify the optimal controller placement. This, along with related inquiries such as the evaluation of election algorithms for the controller in clusters of vehicles, would represent a vital priority for ensuring the success and advancement of this technology.
Finally, this thesis also lays the groundwork for the redesign and upgrade of the device used on the OBU. The patch currently in use to achieve 802.11p functionality is not a long-term solution. Therefore, it would be prudent to explore a more permanent solution, such as SDR.
