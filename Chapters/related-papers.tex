%!TEX root = ../template.tex
%%%%%%%%%%%%%%%%%%%%%%%%%%%%%%%%%%%%%%%%%%%%%%%%%%%%%%%%%%%%%%%%%%%%
%% chapter4.tex
%% NOVA thesis document file
%%
%% Chapter with lots of dummy text
%%%%%%%%%%%%%%%%%%%%%%%%%%%%%%%%%%%%%%%%%%%%%%%%%%%%%%%%%%%%%%%%%%%%
\chapter{Related papers}
\label{cha:related_papers}

% Related Papers

	% Introduction
    Emulation is a convenient and cost-effective method for testing \gls{vanet} solutions. However, due to the complex and unpredictable nature of real-life environments, relying solely on these tools is insufficient\cite{cardona_software-defined_2020}. In order to gain a realistic perspective on \gls{vanet}, it is necessary to implement it in real-world scenarios so the validity of the expected benefits and drawbacks can be tested. The same is true for \gls{sdvn}. 
    \gls{sdvn} is notable for the lack of experimental efforts conducted on actual hardware. In order to enable testing on real scenarios, open-source \gls{sdvn} tools and frameworks that are compatible with a wide range of hardware are indispensable\cite{cardona_software-defined_2020}. This section reviews several related papers that aim to implement \glspl{vanet} and \gls{sdvn} in physical hardware using open sourced components, outlining the key details of their approach and architecture. 

    % VANET
     
    Raviglione et al.\cite{raviglione_open_2019} present a demo paper that assembles an open-source platform based on PC Engines' boards and Unex's \glspl{wnic}. The purpose of this paper is to create a testbed for testing applications that communicate in the vehicular environment. Even though this paper does not attempt to implement \gls{sdn} principles, it is relevant because it provides all the necessary hardware and software components to implement a vehicular testbed.
    The authors assembled two boards consisting of the embedded PC Engines APU1D board with an AMD G-series dual-core T40E x86 \gls{cpu} with 64-bit support and 2 GB \gls{dram}. Communication via 802.11p was achieved using the Unex DHXA-222 \gls{mpcie} card as the \gls{wnic}. This chip is based on the Atheros AR9462 chipset which is supported by the ath9k Linux driver. To enhance storage and memory performance, a \gls{sata} III Transcend MSA370 MCL NAND Flash \gls{ssd} was installed.
    The \gls{os} used was OpenWrt release 18.06.1 with Linux kernel 4.14.63. The authors made modifications to the ath9k Linux driver in order to utilize the channels of the 5.8/5.9 GHz frequency band in accordance with \gls{its}-G5 standards. These modifications were then integrated into the OpenC2X project, which was subsequently ported to OpenWrt. The paper also reviews some modifications and implementations made in higher layers, but these are not relevant to the problem addressed in this thesis.
    
     
    Sedar et al.\cite{sedar_standards-compliant_2021} developed and validated an experimental, standards-compliant \gls{obu}. Their experimental platform is based on an open-source software implementation of the \gls{etsi} \gls{its} protocol stack. Its purpose is to facilitate interoperability in communication between various devices and cloud-based services.
    The \gls{obu} was built using general purpose hardware in the form of a generic laptop, running Ubuntu 18.04. To grant cellular connectivity, it was connected to an \gls{lte} AirPrime EM7565 modem from Sierra Wireless which was in part connected to the 4G cellular network of Vodafone-Spain. The experimental \gls{obu} is also connected to external hardware devices, these being a 4G/5G cellular modem, a \gls{gps}/global navigation satellite system receiver and a connector to receive information from in-vehicle sensors.
    The article presents an overview of the current state of open-source software implementations of the \gls{etsi} \gls{its} protocol stack. It mentions OpenC2X and Vanetza as the two primary implementations. 
    Although real vehicles were not used in testing, the authors state that their experimental platform can be easily integrated into any vehicle.
    
    % SDVN
    
    Secinti et al.\cite{secinti_software_2017} proposed an architectural model that implements \gls{sdn} and virtualization principles in order to enable \gls{vanet} with Wi-Fi access capability. 
    In this architecture, both the \gls{obu} and the \gls{rsu} are implemented using the same type of hardware and software. Both have been implemented using a Raspberry Pi, with the wireless connectivity being provided by the Realtek 5370 Wi-Fi SoC.
    These switches are implemented using OpenvSwitch v2.3.90 running on OpenWRT. OpenvSwitch is a software switch implementation that is open source and natively supported by the Linux kernel\cite{noauthor_open_nodate-2}. Finally, the controller is implemented using OpenDaylight and the southbound \gls{api} used is Openflow.
    
    
    Rito et al.\cite{rito_aveiro_2023} present the deployment and experimentation architecture of the Aveiro Tech City Living Lab in Portugal. The implementation involves a diverse range of devices connected through fiber, radio \gls{its}-G5, and cellular links, utilizing various technologies such as \gls{sdn}, named data networking, and fog computing.
    Vehicle and roadside stations are implemented using the PC Engines APU2 board equipped with an \gls{ssd}, an \gls{ieee} 802.11a/b/g/n \gls{mpcie} wireless card and an \gls{lte} CAT-1 \gls{mpcie} or 5G m.2 module. Additionally, an external USB dual-band wireless adapter was also installed. The operating system used is not specified, but it is a linux distribution because in order to enable the European version of 802.11p in it they used the Linux ath9k driver.
    This paper implements a myriad of different technologies in vehicular infrastructure, one of them being \gls{sdn}. It is notable that it does not implement \gls{sdn} in vehicle \gls{its} stations, but only in the backbone of the network. The purpose of this \gls{sdn} implementation was to use the increased control in the backbone of the vehicular infrastructure and the vehicle information to predict and execute handovers in advance. The authors developed a custom protocol dubbed OBUInfo to provide \gls{cam} information to the controller. This provides the controller with location, heading, speed, and vehicle type information, which is useful in predicting future handovers ahead of time.
    
    
    Sadio et al.\cite{sadio_design_2020} propose a complete \gls{sdvn} prototype design. The hardware used for the \gls{obu} was a Raspberry Pi 3 with Cortex-A53 × 64 1.2 GHz and \gls{sram} 1 GB. This board has access to WiFi 2.4 GHz 802.11 b/g/n and via a Huawei E8372 \gls{lte} USB modem to \gls{lte}. The operating system used was Raspbian Stretch Lite. The authors utilized the Python Twink library to transform the devices into OpenFlow switches. This implementation fails to use the standard for communication in the \gls{vanet} environment, 802.11p.
    \\
    In closing, we present the following table, which provides a concise overview of the pertinent literature.

    \begin{table}[ht]
        \centering
        \begin{tabular}{|p{1.5cm}|p{1.7cm}|p{2cm}|p{1.6cm}|p{1.6cm}|p{1.8cm}|p{2.5cm}|}
        \hline
        \textbf{Paper Reference} & \textbf{Objective} & \textbf{Hardware Used} & \textbf{Software Used} & \textbf{Commu\-nication Protocol} & \textbf{Imple\-mentation Type} & \textbf{Key Contributions/Notes} \\ \hline
        
        Raviglione et al. 2019\cite{raviglione_open_2019} & Create a vehicular testbed & PC Engines APU1D, Unex DHXA-222 \gls{wnic} & OpenWRT 18.06.1, modified ath9k driver & 802.11p & \gls{vanet} & Provided detailed hardware and software to assemble a testbed. \\ \hline
        
        Sedar et al. 2021\cite{sedar_standards-compliant_2021} & Standards-compliant \gls{obu} & Laptop, Sierra Wireless \gls{lte} modem, \gls{gps} receiver & Ubuntu 18.04, open-source \gls{etsi} \gls{its} stack & \gls{lte}, 4G/5G & \gls{vanet} & Used general-purpose hardware, open-source protocol stack. \\ \hline
        
        Secinti et al. 2017\cite{secinti_software_2017} & \gls{sdn}-enabled \gls{vanet} architecture & Raspberry Pi, Realtek 5370 Wi-Fi SoC & OpenWRT, OpenvSwitch v2.3.90, OpenDaylight & Wi-Fi & \gls{sdvn} & Implemented both \gls{obu} and \gls{rsu} using Raspberry Pi with \gls{sdn}. \\ \hline
        
        Rito et al. 2023\cite{rito_aveiro_2023} & Deployment of Aveiro Tech City Living Lab & PC Engines APU2, \gls{mpcie} wireless and \gls{lte} cards & Linux (unspecified), ath9k driver & 802.11p, fiber, \gls{lte}, 5G & Hybrid (\gls{sdn} in backbone only) & Custom protocol (OBUInfo) for handover prediction in \gls{sdn} backbone. \\ \hline
        
        Sadio et al. 2020\cite{sadio_design_2020} & Complete \gls{sdvn} prototype & Raspberry Pi 3, Huawei \gls{lte} USB modem & Raspbian Stretch Lite, Python Twink library & Wi-Fi, \gls{lte} & \gls{sdvn} & Used Raspberry Pi for \gls{obu}, non-standard protocol for \gls{vanet} communication. \\ \hline
        
        \end{tabular}
        \caption{Summary of related literature}
    \end{table}
        