%!TEX root = ../template.tex
%%%%%%%%%%%%%%%%%%%%%%%%%%%%%%%%%%%%%%%%%%%%%%%%%%%%%%%%%%%%%%%%%%%%
%% chapter3.tex
%% NOVA thesis document file
%%
%% Chapter with a short laext tutorial and examples
%%%%%%%%%%%%%%%%%%%%%%%%%%%%%%%%%%%%%%%%%%%%%%%%%%%%%%%%%%%%%%%%%%%%
\chapter{State of Play}
\label{cha:state_of_play}


The plan for this thesis is displayed in the following steps:

\begin{enumerate}
    \item State-of-the-art study
    \begin{enumerate}
        \item In-depth research on SDNs 
        \item In-depth research on VANETs
        \item Investigation into the use of SDNs in vehicular networks
    \end{enumerate}
    \item Familiarization with operating systems 
    \begin{enumerate}
        \item Collection of existing operating systems that are relevant for the problem at hand 
        Linux
        Openwrt
    \end{enumerate}
    \item Testing the chosen operating system on real hardware
    \begin{enumerate}
        \item Installing the OS on a device
        \item Testing the basic operation of an SDN controller in a simple topology
    \end{enumerate}
    \item Study and familiarization with v2x protocol stacks
    \begin{enumerate}
        \item Collection of existing protocol stacks compatible with the chosen operating systems
        OvS
    \end{enumerate}
    \item Testing the chosen protocol stack on real hardware
    \begin{enumerate}
        \item Installing the chosen software in a small device
        \item Testing the software with a simple SDN controller (scenario 1)
    \end{enumerate}
    \item Planning and assembly of an OBU device
    \begin{enumerate}
        \item Analysis and study of the solution's resource consumption 
        \item Choice of hardware capable of supporting the solution
        \item Assembling the device and testing its operation
    \end{enumerate}
    \item Solution feasibility study
    \begin{enumerate}
        \item Researching security vulnerabilities related to device communication with the controller
        \item Study of the efficiency and scalability of the solution
    \end{enumerate}
    \item Writing of the dissertation
\end{enumerate}

Currently the state-of-the art is being developed, with the SDN overview being almost finished.

%Things to test:
% If switch software works
% If remote controller works
% If i can use wireless conection with SDN
% If i can get P4 in the device

A few testing scenarios have also been devised in order to create a gradual complexity increase. 

\begin{enumerate}
    \item Scenario 1:  
    Entities:
    \begin{enumerate}
        \item 2 PCs
        \item 1 switch
    \end{enumerate}
    Establish communication between two PCs with cable over an SDN switch being controlled from the switch itself.
    The controller will program a simple L2 switch. 
    The switch will use OvSwitch.
    \item Scenario 2: 
    Entities:
    \begin{enumerate}
        \item 2 PCs
        \item 1 switch
        \item 1 controller
    \end{enumerate}
    Establish communication between two PCs with cable over an SDN switch being controlled from a remote controller.
    The controller will program a simple L2 switch. 
    \item Scenario 3:  
    Entities:
    \begin{enumerate}
        \item 2 PCs
        \item 1 switch
        \item 1 controller
    \end{enumerate}
    Establish wireless communication between two PCs over an SDN switch being controlled from a remote controller.
    The controller will program a simple L2 switch. 
\end{enumerate}

% estou a meio do estado da arte


% \subsection{Inserting Figures Wrapped with text} % (fold)
% \label{ssec:inserting_images_wrapped_with_text}
% 
% You should only use this feature is \emph{really} necessary. This means, you have a very small image, that will look lonely just with text above and below.
% 
% In this case, you must use the \verb!wrapfiure! package.  To use \verb!wrapfig!, you must first add this to the preamble:
% 
% \begin{wrapfigure}{l}{2.5cm}
%   \centering
%     \includegraphics[width=2cm]{snowman-vectorial}
%   \caption{A snow-man}
% \end{wrapfigure}	
% 
% \noindent\verb!\usepackage{wrapfig}!\\
% This then gives you access to:\\
% \verb!\begin{wrapfigure}[lineheight]{alignment}{width}!\\
% Alignment can normally be either ``l'' for left, or ``r'' for right. Lowercase ``l'' or ``r'' forces the figure to start precisely where specified (and may cause it to run over page breaks), while capital ``L'' or ``R'' allows the figure to float. If you defined your document as twosided, the alignment can also be ``i'' for inside or ``o'' for outside, as well as ``I'' or ``O''. The width is obviously the width of the figure. The example above was introduced with:
% \lstset{language=TeX, morekeywords={\begin,\includegraphics,\caption}, caption=Wrapfig Example, label=lst:latex_example}
% \begin{lstlisting}
% 	\begin{wrapfigure}{l}{2.5cm}
% 	  \centering
% 	    \includegraphics[width=2cm]{snowman-vectorial}
% 	  \caption{A snow-man}
% 	\end{wrapfigure}	
% \end{lstlisting}


%\begin{algorithmic}
%\If {$i\geq maxval$}
%    \State $i\gets 0$
%\Else
%    \If {$i+k\leq maxval$}
%        \State $i\gets i+k$
%    \EndIf
%\EndIf
%\end{algorithmic}

