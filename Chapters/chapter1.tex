%!TEX root = ../template.tex
%%%%%%%%%%%%%%%%%%%%%%%%%%%%%%%%%%%%%%%%%%%%%%%%%%%%%%%%%%%%%%%%%%%
%% chapter1.tex
%% NOVA thesis document file
%%
%% Chapter with introduciton
%%%%%%%%%%%%%%%%%%%%%%%%%%%%%%%%%%%%%%%%%%%%%%%%%%%%%%%%%%%%%%%%%%%

\typeout{NT FILE chapter1.tex}%

\chapter{Introduction}
\label{cha:introduction}

% explain the chapter
This chapter introduces the work presented in this dissertation, beginning with the context and motivation, followed by the main objectives and the methodology used to achieve them. It concludes with a summary of the major contributions and an outline of the document structure.

\section{Context} % (fold)
\label{sec:contex}

% Internet tem imensos benefícios
% Aplicar internet em carros mostra muito potencial
% Nos últimos anos, investigadores têm adaptado a internet para funcionar nestes cenários 
% Isto faz com que estas adaptações sofram com as mesmas limitações da internet (e daí a sua demora)
% SDN resolve alguns problemas da net e tem benefícios
% Trazer esses benefícios para \glspl{vanet} era top
% \glspl{vanet} precisa de ser testada na vida real para ser útil


% The internet is very good
The Internet is universally acknowledged as one of the most significant technological advancements in human history, with its widespread adoption and the numerous benefits it has brought about having irrevocably transformed the manner in which societies in the 21st century live and communicate.

% The internet is not perfect
While this impact is indisputable, the exponential growth observed over the past two decades has revealed significant flaws in Internet design. Indeed, in its current state, the Internet is not fully capable of satisfying the demands of the modern digital landscape. 

% There is huge potential in giving internet to vehicles
One such issue pertinent to this work arises from the fact that our society is becoming increasingly mobile. With the advent of widespread automobile use, researchers and engineers were quick to recognize the vast potential advantages of enabling these vehicles to communicate with one another. Such developments have the potential to enhance a number of aspects of modern transportation, from road security to the quality of life for drivers.
% \glspl{vanet} needs real life testing
Given the direct impact of this technology on driver wellbeing, it is crucial that rigorous real-world testing be conducted to guarantee the safety of the general public.

% Researchers have been adapting the internet for the vehicular space for years
For the past three decades, universities, companies, and governments have been engaged in collaborative efforts to develop the technology known as \glspl{vanet}, whose objective is to enable vehicles to communicate directly with each other on the road.
% These adaptations inherit problems of internet design
This process has made enormous progress in adapting the Internet to vehicular environments. %Nevertheless, these solutions retain some of the aforementioned shortcomings inherent in Internet designs.

% What SDN is and why it is good
\gls{sdn}, a new networking paradigm, promises to deliver increased innovation, reduced cost, simplified management, and improved performance.
% SDN helps to solve may of the problems of the internet
It thus promises to address the limitations of the Internet and facilitate further growth. In light of this potential, researchers have recently focused their attention on this promising field, treating it as a potential solution to many of the challenges currently faced in the current Internet landscape.
% SDN’s advantages could help VANETS
Notably, this has occurred in the domain of \gls{vanet}, which has led to the creation of the \gls{sdvn} field. The potential for \gls{sdn} to enhance the capabilities of this technology has prompted recent studies to apply \gls{sdn} to vehicular networks with the goal of developing optimized solutions for the future. The capabilities of \gls{sdvn} are not proven to be superior to regular \glspl{vanet}, but this thesis assumes that the application of \gls{sdn} principles in \glspl{vanet} is worthwhile.

% section contex (end)

\section{Motivation} % (fold)
\label{sec:motivation}

% \gls{sdvn}s precisa de testes na vida real e não em simulador para a sua viabilidade ser testada

In accordance with the indispensable role that real-world testing plays in the ultimate success of \gls{vanet}, it is of the utmost importance to conduct comprehensive testing in realistic settings to ensure the successful deployment of \gls{sdvn}. 

The majority of \gls{sdvn} projects are designed for use in the real world, yet they are predominantly subjected to simulation-based testing. Although the outcomes of simulations are of great importance, they must undergo a process of confirmation and validation to ensure their reliability. The most effective method for achieving this validation is to compare the simulation results with outcomes derived from relevant, real-world testing environments. This emphasizes the pressing need for the availability of real hardware testing platforms to achieve more accurate and reliable results in this critically important area. 

The development of a functional hardware testing platform, utilizing free and open-source software components for \gls{sdvn} projects, represents a substantial advancement in the evolution of both the \gls{sdn} and \gls{vanet} fields. This may demonstrate the value of \gls{sdn} beyond the data center, establishing it as a dominant force in the field while also fostering improvements in \gls{vanet}.

% section motivation (end)

\section{Objectives} % (fold)
\label{sec:objectives}

% Build SDVN device
This dissertation presents a proposal for the design and implementation of a functional \gls{sdvn} device. In essence, the core objective is to develop a device that can be classified as both a \gls{sdn} and a \gls{vanet} device in a single entity. 

% device will be an OBU
As a further clarification, the device will be designed and assembled to fulfill the role of an \gls{obu} and thus be able to communicate in the  vehicular ad hoc space. The device must also conform to the design principles of \gls{sdn}.

This document will undertake a detailed and thorough analysis of the \gls{sdn} and \gls{vanet} fields in order to define the characteristics of this device. Nevertheless, it is possible to establish a preliminary set of guidelines to which the device must adhere even before such an evaluation is conducted. The following two criteria serve as the fundamental basis for the evaluation of the devices in question. First and foremost, the device must be affordable. Secondly, the software components utilized must be open-source, wherever feasible.

% section objectives (end)

\section{Methodology} % (fold)
\label{sec:methodology}

% Introduction
The methodology employed in this work was designed with the objective of assembling the necessary theoretical basis for an assessment of the viability of the \gls{sdvn} technology as a whole. The selected methodologies are explained and justified in the context of the research objectives.

% Primeiro começar por estudar as duas áreas e as suas nuances
The dissertation commences with a comprehensive analysis of the \gls{sdvn} field. In order to achieve the proposed objective of developing an \gls{sdn}-compatible \gls{obu}, it is essential to conduct such research. While a review of the existing software in the \gls{sdvn} field is undoubtedly important, the goal of the research step of this document is to extensively analyze the \gls{sdn} and \gls{vanet} fields.

% Estudar SDN, a sua motivação, vantagens e software usado
In regard to \gls{sdn}, this research aims to provide a rationale for its integration with \gls{vanet}. It is therefore essential to present a comprehensive analysis of the underlying motivations, advantages, and fundamental principles that shape this technology.

% Estudar \glspl{vanet}, as suas características e problemas, o ponto em que está esta tecnologia (a.k.a todo o trabalho de implementação que foi feito até agora) e software existente
Conversely, research on \gls{vanet} is primarily concerned with defining the characteristics that distinguish this field of study from other ad hoc technologies, thereby establishing a comprehensive understanding of this field. Moreover, it is of critical importance to gain insight into the current state of this technology and to identify the remaining steps required for its widespread adoption. In light of this, this study is primarily focused on investigating the advancements in the field within the European context.

% Estudar SDVN e trabalhos existentes de implementação em hardware real
The research component of this document concludes with a review of existing literature, the objective of which is to identify any studies that share similar motivations to this document.

% Tentar construir um dispositivo SDN que funcione em ambientes veiculares e seja o máximo open-source possível
The practical component of this dissertation is based on with a period of rigorous experimentation and familiarization with existing software in the field. Subsequently, a new \gls{sdn} \gls{obu} is designed and assembled, utilizing upgraded hardware and adhering to the requisite standards for \gls{sdn} compliance.

% Testar o dispositivo em vários cenários reais
To conclude, a concrete testing scenario is defined and subsequently executed with the intention of illustrating and validating the aforementioned achievements.


% section methodology (end)

\section{Main contributions} % (fold) 
\label{sec:main_contributions}

The primary contributions of this dissertation are as follows:
\begin{itemize}
    \item A comprehensive and insightful analysis of the regulations governing \glspl{vanet}.
    \item A detailed examination of \gls{sdn} and its underlying motivations and benefits.
    \item A well-founded and realistic assessment of the potential implications of \gls{sdvn} for future developments.
    \item An evaluation of hardware and software alternatives for a \gls{sdn} capable \gls{obu} prototype.
    \item Creation of a functional \gls{obu} prototype with \gls{sdn} support.
    \item The evaluation of the prototye in predefined scenarios.

\end{itemize}
These achievements are significant as they demonstrate in practice how it is possible to build and test a \gls{v2x} device with \gls{sdn} and \gls{p4} support, laying the groundwork for new experiments and advancements in the field of \gls{sdvn}.

% section main_contributions (end)

\section{Document structure} % (fold)
\label{sec:document_structure}

This document is divided into eight other main chapters. Chapter~\ref{cha:vehicular_networks} provides a broad context about vehicular networks, in order to understand the main challenges in this field. In chapter~\ref{cha:sdn}, a very detailed overview of \gls{sdn} is made, with a higher emphasis on the data plane, to show the power and potential of this technology. Chapter~\ref{cha:sdvn} addresses the advantages and difficulties of using \gls{sdn} in a vehicular context, while also presenting the \gls{sdvn} architecture. It also presents the studies in this field that share a similar objective to the present study are reviewed and analyzed. Chapter~\ref{cha:device_design} details the implementation phase of this dissertation by delineating the design of the prototype in layers and concluding with a description of the hardware that will be utilized. In chapter~\ref{cha:software_tests}, the pertinent software and hardware are evaluated, showing that the complete prototype proposed operates as planned. Chapter~\ref{cha:conclusion} concludes this document and outlines future work.


% section document_structure (end)

