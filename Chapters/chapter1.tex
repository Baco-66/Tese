%!TEX root = ../template.tex
%%%%%%%%%%%%%%%%%%%%%%%%%%%%%%%%%%%%%%%%%%%%%%%%%%%%%%%%%%%%%%%%%%%
%% chapter1.tex
%% NOVA thesis document file
%%
%% Chapter with introduciton
%%%%%%%%%%%%%%%%%%%%%%%%%%%%%%%%%%%%%%%%%%%%%%%%%%%%%%%%%%%%%%%%%%%

\typeout{NT FILE chapter1.tex}%

\chapter{Introduction}
\label{cha:introduction}

% explain the chapter
This chapter serves as an introduction to this thesis, providing an overview of its context, motivation, objectives, and methodology.

\section{Contextualization} % (fold)
\label{sec:contex}

% The internet is very good
The Internet is universally acknowledged as one of the most significant technological advancements in human history, with its widespread adoption and the numerous benefits it has brought about having irrevocably transformed the manner in which societies in the 21st century live and communicate. 

% The internet is very static and need to grow for portable scenarios
Despite the Internet's numerous strengths, it is evident that, in its current form, it is not fully capable of satisfying the demands of the modern digital landscape. One such problem relevant to this thesys stems from the fact that our society is becoming increasingly mobile. With the widespread adoption of automobiles, researchers and engineers soon realised that allowing these these vehicles to comunicate with each other could offer major benefits to road security and could even bring quality of life improvements to drivers. % Não é correto dizer que a internet não esta preparada porque já esta basicamente tudo feito para a comunicação funcionar

% What SDN is and why it is good
A new networking paradigm, software defined networks, promises to offer more innovation, reduced cost, simplify management and improve performance. SDN is poised to address the limitations of the Internet and facilitate further growth.

% SDN helps to solve may of the problems of the internet, and this one could be one of them as well
As such, in recent years, researchers have turned to this new and exciting field, with SDN promising a solution to many of the challenges faced by the current internet landscape.Therefore, a lot of recent studies have applied SDNs to vehicular networks in order to create better solutions for the future.
This thesis aims to introduce SDN principles to V2X communication.

% section contex (end)

\section{Motivation} % (fold)
\label{sec:motivation}

%Assim sendo, existem muitas aplicações e casos de estudo sobre SDNs em redes veiculares. Porem, a sua maioria utiliza simuladores para testar o seu funcionamento, pelo que se nota a necessidade da existência de dispositivos OBUs cite{hardware} cite{rasberypi} para realizar testes mais fidedignos. 

In recent years, there has been a growing interest in the application of SDN technology in vehicular networks, with a large number of projects in SDVNs being created. Even doe these projects aim to be used in the real world, the large majority of them is restricted to simulation-based testing which reduces the veracity of the results. Therefore, this highlights a pressing need for the availability of real hardware testing platforms, in order to achieve more accurate and reliable results in such a critically important area. The development of a real hardware testing platform for SDVN projects will allow for a more comprehensive evaluation of the technology, which is a step further for this technology to become accepted and used.

% section motivation (end)

\section{Objectives} % (fold)
\label{sec:objectives}

Our main goal in this thesis is to design and assemble a hardware device similar to an OBU, utilizing open-source software where possible. This device must be SND compatible, so its routing tables must be configurable by a local or remote SDN controller. To meet these requirements, the device will run an open-source V2X protocol stack on a Linux-based OS.

%Neste trabalho pretende-se dotar um dispositivo OBU real, a correr uma imagem Linux com pilha protocolar V2X open-source, de capacidades SDN. #@# O objetivo é que o referido OBU (a ser fornecido) possa ser configurado por um controlador SDN externo usando um protocolo standard e também open-source, como por exemplo o OpenFlow. Para tal será necessário preparar e configurar devidamente a imagem do software do OBU com suporte SDN, que possa ser testada num cenário real simples (a definir).
%Adicionalmente será estudada uma nova versão de um OBU com especificação de hardware e de software com vista a novos testes futuros.

%No final desta dissertação espera-se conseguir um dispositivo OBU real com capacidade de comunicação multi-tecnologia (nomeadamente LTE/4G/5G, Wifi, IEEE 802.11p) e suporte SDN. Adicionalmente, o dispositivo deverá ser testado num cenário experimental de modo a obter resultados reais.


% objective is to build an obu

% what is an obu (ja devo dizer isto na descrição das VANETs e SDNVs)

% what are the theoretical objectives it has to achieve 
    % cheap
    % SND compatible
    % open-source
    %

% what are its requirements
    % communication is multi-technology compatible
    %

% objectives divided in two phases: 
    % program the softawre of a SDVN OBU,
    % upgrade SDVN OBU software
    %


% section objectives (end)

\section{Methodology} % (fold)
\label{sec:methodology}


% Numa fase inicial, será feita uma exploração sobre o software necessário para o funcionamento de um OBU controlado por um controlador ryu. Posteriormente, será projetado e construido um novo dispositivo OBU com hardware moderno. Por fim, será testado performance e vulnerabilidades de segurança.

Initially, an investigation will be done to find the software required for the implementation of an OBU controlled by a free controller. Subsequently, a new OBU device will be designed and assembled, using modern hardware. Finally, performance and security vulnerabilities will both be tested.

% section methodology (end)

%\section{Main contributions} % (fold) 
%\label{sec:main_contributions}

% section main_contributions (end)

\section{Document structure} % (fold)
\label{sec:document_structure}

Chapter 2 is divided into three main sections. Section 2.1. aims to provide a broad context about vehicular networks, in order to understand the main challenges in this field. In section 2.2., a very detailed overview of SDN is made, with a large emphasis on the data and control plane, to show the power and potential of this technology. Section 2.3. serves to address what are the advantages and shortcomings of using SDN in a vehicular context. 
In Chapter 3, a state of play is made, where the plan to achieve the goals of this thesis is highlighted. 

% section document_structure (end)

