%!TEX root = ../template.tex
%%%%%%%%%%%%%%%%%%%%%%%%%%%%%%%%%%%%%%%%%%%%%%%%%%%%%%%%%%%%%%%%%%%%
%% implementation.tex
%% NOVA thesis document file
%%
%% Chapter with lots of dummy text
%%%%%%%%%%%%%%%%%%%%%%%%%%%%%%%%%%%%%%%%%%%%%%%%%%%%%%%%%%%%%%%%%%%%
\chapter{Functional Tests}
\label{cha:software_tests}
% Mudar tudo para functional tests

% Software Tests
% Introduction
In this chapter, the selected hardware prototype is tested, in conjunction with the desired software, and te results presented. The objective of these tests is to ascertain the viability of the software selected for practical deployment of the prototypes. Specifically, this study sought to evaluate the viability of both \gls{ovs} and \gls{bmv2} in vehicular scenarios. In particular, the aim of the forthcoming testing is to ascertain the degree of compatibility between the aforementioned switching emulation software and a number of other components, including the ath9k patch, the \gls{onos} controller, and the hardware of the device in question.


% Experimental Setup
\section{Experimental Setup}
In consideration of the practical implications of the goals of this dissertation, the device must meet the criteria for an \gls{obu}/Vehicle \gls{its} sub-system, in accordance with the \gls{sdn} paradigm. For this, the device must be categorized as both \gls{sdn} and \glspl{vanet}. In practical terms, for a device to be considered \gls{sdn}-compliant, it is necessary to utilize software to emulate the functions of an \gls{sdn} device. Moreover, the minimum requirement for a device to be appropriate for the context of \glspl{vanet} is its capability of establishing connectivity in accordance with the \gls{ieee} 802.11p standard.
Accordingly, the study employs a three-stage experimental methodology, consisting of installation, integration, and functionality verification. The tests are performed in a series of phases, with each subsequent phase building upon the preceding one and introducing greater degrees of complexity. Their objective is to assess the compatibility of the various elements utilized, namely the hardware, the ath9k patch, and \gls{ovs}/\gls{bmv2}. The success of this endeavor is contingent upon the establishment of a connection between devices, which is conducted using the ping tool.
The experiments were conducted on the devices described in Section~\ref{sec:box_v1} and~\ref{sec:box_v2} with the assistance of an external device, namely a personal computer. Additionally, a switch and six network cables were employed to facilitate communication between the devices and the personal computer, while omnidirectional antennas operating at 6GHz were utilized to enable communication in the requisite frequency range.
The software tools utilized in this investigation were as follows:
\begin{itemize}
	\item Windows 10 is the operating system of the \gls{pc};
	\item Oracle VM Virtualbox 7.0.4 was installed on the host computer and employed to both install and run the \gls{onos} and p4c;
	\item \gls{onos} 2.7.0 was utilized for the controller;
	\item p4c is the compiler for the \gls{p4} language;
	\item PuTTy Release 0.80 was installed on the \gls{pc} and is used to interface with the machines;
	\item Ubuntu 24.04.1 LTS operating system is used on devices version 2;
	\item Debian 12 operating system is used on device version 1;
	\item \gls{ovs} and \gls{bmv2} are the focus of the current testing.
\end{itemize}
It is essential to acknowledge that, despite the comprehensive and rigorous nature of these tests, they are not without inherent limitations. Such limitations include the specific conditions of the controller environment and the exclusion of real-world network conditions. The methodology selected was chosen for its capacity to facilitate the systematic identification of critical compatibility issues that must be addressed for successful real-world deployment. Moreover, for the purposes of this study, it was assumed that the devices lacked the requisite resources to install and run \gls{onos}.


% OvS
\section[Experiment 1: testing the integration of OvS]{Experiment 1: testing the integration of \gls{ovs}}
% Test Scenarios
This experiment is intended to assess the compatibility and practicality of integrating \gls{ovs} and OpenFlow with \gls{vanet} technologies. The initial stages of the experiment were conducted using a single device. Around the midpoint of experimentation, the additional two devices of version 2 arrived and were incorporated into the test setup.

			% Phase 1
\subsection[Testing phase 1: standalone device with OvS and local controller]{Testing phase 1: standalone device with \gls{ovs} and local controller}
The opening phase of the experiment sought to examine the functionality of \gls{ovs} in a relatively straightforward scenario, with the additional objective of becoming familiar with the technology. Accordingly, the procedure commenced with a standalone examination of \gls{ovs}, which is possible due to the fact that \gls{ovs} is equipped with a local controller that is capable of autonomously managing network traffic. \gls{ovs} was installed on the device, and a virtual network was created within the device using \glspl{vrf} and virtual interfaces.

The setup from this phase can be observed in Figure~\ref{fig:exp1_phase1_setup}, and is diagrammatically represented in Figure~\ref{fig:exp1_phase1_diagram}.
When viewed through the lens of \gls{sdn}, the configuration of this phase can be conceptualized as illustrated in Figure~\ref{fig:exp1_phase1_sdn_diagram}.

\begin{figure}
	\centering
	\includegraphics[width=0.5\textwidth]{Chapters/Figures/tests/ovs_phase_1/20241122_155830.jpg}
	\caption{Setup from the first phase of the experiments with \gls{ovs}}
	\label{fig:exp1_phase1_setup}
\end{figure}

\begin{figure}
	\centering
	\includegraphics[width=0.6\textwidth]{Chapters/Figures/tests/ovs_phase_1/setup_diagram.PNG}
	\caption{Diagram that represents the setup from the first phase of the experiments with \gls{ovs}}
	\label{fig:exp1_phase1_diagram}
\end{figure}

\begin{figure}
	\centering
	\includegraphics[width=0.6\textwidth]{Chapters/Figures/tests/ovs_phase_1/sdn_diagram.PNG}
	\caption{Diagram that depicts the \gls{sdn} perspective of the setup from the first phase of the experiments with \gls{ovs}}
	\label{fig:exp1_phase1_sdn_diagram}
\end{figure}


Figure~\ref{fig:exp1_phase1_ovsctl_show} presents the configuration of \gls{ovs}, along with information regarding the device interfaces. Here, all valid interfaces on the device are displayed. This list includes two virtual interfaces, which can be seen to have been added to the \gls{ovs} bridge.
Figure~\ref{fig:exp1_phase1_pings} demonstrates the existing interfaces of the two \glspl{vrf} created in this device, which are shown to be on the same subnet. Following this, a ping is executed between the two devices to demonstrate their ability to communicate.

\begin{figure}
	\centering
	\includegraphics[width=\textwidth]{Chapters/Figures/tests/ovs_phase_1/device_ports&ovs_config.PNG}
	\caption{Results from running the commands "ip a" and "ovs-vsctl show" in phase 1}
	\label{fig:exp1_phase1_ovsctl_show}
\end{figure}

\begin{figure}
	\centering
	\includegraphics[width=\textwidth]{Chapters/Figures/tests/ovs_phase_1/VRF_config_&_pings.PNG}
	\caption{Results from running the commands "ip a" on both \glspl{vrf} and "ping" between them in phase 1}
	\label{fig:exp1_phase1_pings}
\end{figure}

			% Phase 2
\subsection[Testing phase 2: standalone device with OvS and remote ccontroller]{Testing phase 2: standalone device with \gls{ovs} and remote ccontroller}
Following the success of this initial phase, the decision was made to proceed with the introduction of the remote controller. The controller, which will be utilized for the remainder of the tests, introduced an elevated level of complexity, providing an excellent opportunity to become acquainted with \gls{onos} environments. Consequently, phase two entailed the creation of a Linux-based \gls{vm} utilizing VirtualBox, with the installation of \gls{onos} on this \gls{vm}. \gls{onos} was therefore incorporated into the preceding scenario and connected to \gls{ovs} via OpenFlow.

The setup from this second phase is visualy the same as in the first phase, and can therefore be observed in Figure~\ref{fig:exp1_phase1_setup}. The setup of this phase is diagrammatically represented in Figure~\ref{fig:exp1_phase2_diagram}. When viewed through the lens of \gls{sdn}, the configuration of this phase can be conceptualized as illustrated in Figure~\ref{fig:exp1_phase2_sdn_diagram}.

\begin{figure}
	\centering
	\includegraphics[width=0.7\textwidth]{Chapters/Figures/tests/ovs_phase_2/setup_diagram.PNG}
	\caption{Diagram that represents the setup from the second phase of the experiments with \gls{ovs}}
	\label{fig:exp1_phase2_diagram}
\end{figure}

\begin{figure}
	\centering
	\includegraphics[width=0.6\textwidth]{Chapters/Figures/tests/ovs_phase_2/sdn_diagram.PNG}
	\caption{Diagram that depicts the \gls{sdn} perspective of the setup from the second phase of the experiments with \gls{ovs}}
	\label{fig:exp1_phase2_sdn_diagram}
\end{figure}

The configuration of \gls{ovs}, in conjunction with the demonstration of connectivity between the two \glspl{vrf}, is illustrated in Figure~\ref{fig:exp1_phase2_pings}. This particular \gls{ovs} state is identical to that of the phase 1 of testing.

\begin{figure}
	\centering
	\includegraphics[width=0.8\textwidth]{Chapters/Figures/tests/ovs_phase_2/ovs_config_&_pings.PNG}
	\caption{Results from running the commands "ovs-vsctl show" and "ping" between both \glspl{vrf} in phase 2}
	\label{fig:exp1_phase2_pings}
\end{figure}

Figure~\ref{fig:exp1_phase2_onos} displays a selection of relevant commands that were executed in the \gls{onos} console.
This output demonstrates that \gls{onos} is connected to one device operating \gls{ovs} via OpenFlow version 1.3, and has knowledge of two hosts. Furthermore, we observe the flows that have been automatically integrated by the \gls{onos} services, thereby providing the switch with the functionality to forward packets.

\begin{figure}
	\centering
	\includegraphics[width=\textwidth]{Chapters/Figures/tests/ovs_phase_2/onos_topology_&_more.PNG}
	\caption{\gls{onos} console with information relating to the devices, hosts and flows of the network in phase 2}
	\label{fig:exp1_phase2_onos}
\end{figure}

			% Phase 3
\subsection[Testing phase 3: multiple devices with OvS and remote controller]{Testing phase 3: multiple devices with \gls{ovs} and remote controller}
By the time phase 3 was reached, the two additional devices had been received. Therefore, the test \gls{sdn} network was expanded to integrate these twp additional devices. \gls{ovs} was installed on the two new devices, and the three devices were linked together and connected to the \gls{onos} controller, and two \glspl{vrf} were created, each situated on opposite edges of the network. 

The configuration of this phase is illustrated in Figure~\ref{fig:exp1_phase3_setup}. The setup of this phase is diagrammatically represented in Figure~\ref{fig:exp1_phase4_diagram} and described in the following itemized list:

\begin{itemize}
	\item 192.168.137.1 - This \gls{ip} address identifies the author's personal laptop computer. It serves as the central coordinating hub for the entire network, facilitating the observation and monitoring of all network activity.
	\item 192.168.137.2 - This address represents a virtual machine that is hosted on the author's laptop via the Oracle VM VirtualBox virtualization platform. The virtual environment is configured with Debian 12 and is primarily intended for the deployment of \gls{onos}.
	\item 192.168.137.252 - This corresponds to a device version 2. It has a fresh installation of the Ubuntu Server operating system and is running \gls{ovs}.
	\item 192.168.137.253 - This device is virtually identical to the device 192.168.137.252.
	\item 192.168.137.254 - This address represents device version 1. This device has debian 12 and is running \gls{ovs}.
	\item 10.10.10.1 - This \gls{ip} address identifies the virtual interface that has been added to the \gls{vrf} instance created in the 192.168.137.252 device.
	\item 10.10.10.2 - Identically to the last one, this \gls{ip} address identifies the virtual interface that has been added to the \gls{vrf} instance created in the 192.168.137.253 device.
\end{itemize}

\begin{figure}
	\centering
	\includegraphics[width=0.5\textwidth]{Chapters/Figures/tests/ovs_phase_3/setup.jpg}
	\caption{Setup from the third phase of the experiments with \gls{ovs}}
	\label{fig:exp1_phase3_setup}
\end{figure}

\begin{figure}
	\centering
	\includegraphics[width=0.8\textwidth]{Chapters/Figures/tests/ovs_phase_4/setup_diagram.PNG}
	\caption{Diagram that represents the setup from the third phase of the experiments with \gls{ovs}}
	\label{fig:exp1_phase4_diagram}
\end{figure}

			% Results

When viewed through the lens of \gls{sdn}, the configuration of this phase can be conceptualized as illustrated in Figure~\ref{fig:exp1_phase4_sdn_diagram}.

\begin{figure}
	\centering
	\includegraphics[width=0.7\textwidth]{Chapters/Figures/tests/ovs_phase_4/sdn_diagram.PNG}
	\caption{Diagram that depicts the \gls{sdn} perspective of the setup from the third phase of the experiments with \gls{ovs}}
	\label{fig:exp1_phase4_sdn_diagram}
\end{figure}



The configuration of the three instances of \gls{ovs} that are running is Figure~\ref{fig:exp1_phase4_config}. This figure presents the configurations of the three instances of OvS installed on each individual device. These three instances of \gls{ovs} are also acknowledged by the controller, as illustrated in Figure~\ref{fig:exp1_phase4_onos}. Once again, these devices are connected to the controller via OpenFlow version 1.3, and the controller is aware of two hosts on the network. 
Figure~\ref{fig:exp1_phase4_onos_flows} presents the flows that were injected into each device. These flows are identical across all three devices and are consistent with the flows injected in the previous phase, due to the fact that \gls{onos} has precisely the same services activated.
Connectivity between the two \gls{vrf} is demonstraded in Figure~\ref{fig:exp1_phase4_pings}. It illustrates the connectivity between devices 192.168.137.252 and 192.168.137.253, manifesting through the VFRs with \glspl{ip} 10.10.10.1 and 10.10.10.2.

\begin{figure}
	\centering
	\includegraphics[width=\textwidth]{Chapters/Figures/tests/ovs_phase_4/ovs_config.PNG}
	\caption{Results from running the commands "ovs-vsctl show" in the devices running \gls{ovs} in phase 3}
	\label{fig:exp1_phase4_config}
\end{figure}

\begin{figure}
	\centering
	\includegraphics[width=\textwidth]{Chapters/Figures/tests/ovs_phase_4/onos_topology.PNG}
	\caption{\gls{onos} console with information relating to the devices and hosts of the network in phase 3}
	\label{fig:exp1_phase4_onos}
\end{figure}

\begin{figure}
	\centering
	\includegraphics[width=\textwidth]{Chapters/Figures/tests/ovs_phase_4/onos_flows.PNG}
	\caption{\gls{onos} console with information relating to the flows of the network in phase 3}
	\label{fig:exp1_phase4_onos_flows}
\end{figure}

\begin{figure}
	\centering
	\includegraphics[width=\textwidth]{Chapters/Figures/tests/ovs_phase_4/pings.PNG}
	\caption{Results from running the commands "ping" in the \gls{vrf} between the two devices on the edge of the network in phase 3}
	\label{fig:exp1_phase4_pings}
\end{figure}

 		% Analysis and Interpretation
\subsection{Analysis and Interpretation}
It was intended for the subsequent phase of the investigation to involve an attempt to introduce the ath9k patch and wireless antennas for use in conjunction with the \gls{ovs} system. However, Phase 3 was the final phase of this examination. Through additional research it was discovered that \gls{ovs} is unable to function with wireless interfaces, which is corroborated by a statement on the official wiki~\cite{noauthor_common_nodate} indicating that \gls{ovs} is not compatible with wireless interfaces.
Furthermore, the testing phase revealed another significant limitation in the capabilities of \gls{ovs}. Despite its compatibility with OpenFlow, \gls{ovs} is constrained by the traditional definition of a switch, thereby limiting its functionality to layer 2. This inherent restriction disables the system from parsing packets from different IP subets.
In conclusion, it was demonstrated that \gls{ovs} is not ready to be used within the context of vehicular scenarios. 

% BMv2
\section[Experiment 2: testing the integration of BMv2 and P4]{Experiment 2: testing the integration of \gls{bmv2} and \gls{p4}}
% Test Scenarios
The objective of this experiment is to evaluate the compatibility and practicality of integrating \gls{bmv2} and \gls{p4} with \gls{vanet} technologies. The experiments were conducted using two 2-box apparatuses.

% Phase 1
\subsection[Testing phase 1: P4 dataplane on devices]{Testing phase 1: \gls{p4} dataplane on devices}
% Or Testing phase 1: multiple devices with BMv2 and P4

Phase 1 of this experiment will focus on assessing the fundamental operational capabilities of \gls{bmv2}, thereby offering an opportunity for familiarization with this environment. In order to accomplish this goal, the first step is installing \gls{bmv2} on the target devices. It became evident that this process was more intricate than initially anticipated, due to the fact the southbound \gls{api} of \gls{bmv2} is, by default, not P4Runtime, but rather Thrift~\cite{noauthor_apache_nodate}. Following an investigation into the matter, the existence of an option to install supplementary components to \gls{bmv2} was identified, thereby enabling communication via the GRPC-based P4Runtime. The modification was implemented, allowing for the establishment of a connection between the devices and \gls{onos}.

The \gls{p4}/\gls{onos} application developed in section \ref{sec:code} was used. \gls{p4} code is compiled by the p4c~\cite{noauthor_p4langp4c_nodate}, the code is automatically injected into the devices by \gls{onos}, thereby eliminating the need for manual transfer of the file on each occasion.
Once the requisite setup was completed, two \glspl{vrf} were created, one on each device. Subsequently, the aforementioned \gls{vrf} instances were linked to their corresponding devices via the use of virtual interfaces. Thereafter, \gls{bmv2} was initiated, with one virtual and one physical interface allocated to it. Next, \gls{onos} was activated, along with the requisite \gls{onos}/\gls{p4} application and all associated dependencies. At last, the configuration was applied to the controller, thereby informing the controller of the devices and providing it with the necessary information to perform its functions.

The test configuration is illustrated in Figure~\ref{fig:exp2_phase1_setup}, which was also used as the basis for the diagram seen in Figure~\ref{fig:exp2_phase1_diagram}. A description of said diagram is in the following itemized list:
\begin{itemize}
	\item 192.168.137.1 - This \gls{ip} address identifies the author's personal laptop computer. It serves as the central coordinating hub for the entire network, facilitating the observation and monitoring of all network activity.
	\item 192.168.137.2 - This address represents a virtual machine that is hosted on the author's laptop via the Oracle VM VirtualBox virtualization platform. The virtual environment is configured with Debian 12 and is primarily intended for the deployment of \gls{onos}.
	\item 192.168.137.252 - This corresponds to a device version 2. It has a fresh installation of the Ubuntu Server operating system and is running \gls{bmv2} with grpc.
	\item 192.168.137.253 - This device is virtually identical to the device 192.168.137.252.
	\item 10.10.10.1 - This \gls{ip} address identifies the virtual interface that has been added to the \gls{vrf} instance created in the 192.168.137.252 device.
	\item 10.10.10.2 - Identically to the last one, this \gls{ip} address identifies the virtual interface that has been added to the \gls{vrf} instance created in the 192.168.137.253 device.
\end{itemize}

\begin{figure}
	\centering
	\includegraphics[width=0.5\textwidth]{Chapters/Figures/tests/bmv2_phase_1/20241122_185833.jpg}
	\caption{Setup from the first phase of the experiments with \gls{bmv2}}
	\label{fig:exp2_phase1_setup}
\end{figure}

\begin{figure}
	\centering
	\includegraphics[width=0.6\textwidth]{Chapters/Figures/tests/bmv2_phase_1/setup_diagram.PNG}
	\caption{Diagram that represents the setup from the first phase of the experiments with \gls{bmv2}}
	\label{fig:exp2_phase1_diagram}
\end{figure}

When viewed through the lens of \gls{sdn}, the physical configuration can be conceptualized as illustrated in Figure~\ref{fig:exp2_phase1_sdn_diagram}.

\begin{figure}
	\centering
	\includegraphics[width=0.7\textwidth]{Chapters/Figures/tests/bmv2_phase_1/sdn_diagram.PNG}
	\caption{Diagram that depicts the \gls{sdn} perspective of the setup from the first phase of the experiments with \gls{bmv2} in phase 1}
	\label{fig:exp2_phase1_sdn_diagram}
\end{figure}


Figure~\ref{fig:exp2_phase1_bmv2} shows \gls{bmv2} running in both devices. It displays the initial debug messages that were transmitted by the program in order to create the P4 tables.

\begin{figure}
	\centering
	\includegraphics[width=\textwidth]{Chapters/Figures/tests/bmv2_phase_1/bmv2_running.PNG}
	\caption{Consoles running \gls{bmv2} on both devices}
	\label{fig:exp2_phase1_bmv2}
\end{figure}

The list of devices and hosts known to the controller is shown in Figure~\ref{fig:exp2_phase1_onos}, comprising two devices and two hosts. The devices are interfaced with the controller via P4Runtime. The list of applications activated in \gls{onos} is presented in Figure~\ref{fig:exp2_phase1_onos_apps}. This list includes several applications that have been activated to enable the P4runtime to function with \gls{bmv2}. Furthermore, the final application listed, designated "SDVN Test App", is the \gls{p4}/\gls{onos} application developed for this phase.
The flows that have been injected into the two devices are illustrated in Figure~\ref{fig:exp2_phase1_onos_flows}, and as expected, these flows are identical on both devices.
It is also possible to identify the table in which each flow is applied. The flows designated for the "acl\_table" are inserted by \gls{onos} native applications. In contrast, the other three rules, injected into the "l2\_exact\_table", are all inserted by our application. These include a default traffic rule, a broadcast rule and an unicast rule.

\begin{figure}
	\centering
	\includegraphics[width=\textwidth]{Chapters/Figures/tests/bmv2_phase_1/onos_topology.PNG}
	\caption{\gls{onos} console with information relating to the devices and hosts of the network in phase 1}
	\label{fig:exp2_phase1_onos}
\end{figure}

\begin{figure}
	\centering
	\includegraphics[width=\textwidth]{Chapters/Figures/tests/bmv2_phase_1/onos_apps.PNG}
	\caption{\gls{onos} console with information relating to the applications activated on the controller in phase 1}
	\label{fig:exp2_phase1_onos_apps}
\end{figure}

\begin{figure}
	\centering
	\includegraphics[width=\textwidth]{Chapters/Figures/tests/bmv2_phase_1/flows.PNG}
	\caption{\gls{onos} console with information relating to the flows of the network in phase 1}
	\label{fig:exp2_phase1_onos_flows}
\end{figure}

Figure~\ref{fig:exp2_phase1_pings} illustrates the connectivity between devices 192.168.137.252 and 192.168.137.253, manifesting through the \glspl{vrf} with \glspl{ip} 10.10.10.1 and 10.10.10.2.

\begin{figure}
	\centering
	\includegraphics[width=\textwidth]{Chapters/Figures/tests/bmv2_phase_1/pings-2.PNG}
	\caption{Results from running the commands "ping" in the \gls{vrf} between the two devices on the network in phase 1}
	\label{fig:exp2_phase1_pings}
\end{figure}

% Phase 2
\subsection[Testing phase 2: P4 control over wireless interfaces]{Testing phase 2: \gls{p4} control over wireless interfaces}
% Testing phase 2: BMv2 and P4 Dataplane Integration for Wireless Communications
% Testing phase 2: BMv2 Devices with P4 Dataplane over a Wireless Path

Phase two of the experiments with \gls{bmv2} represents a significant advancement over the initial phase, as it introduces the crucial element of wireless communication. Accordingly, the ath9k patch was implemented on the devices to enable communication at the \gls{ieee} 802.11p standard. Figure~\ref{fig:exp2_phase2_wireless} displays the lists of interfaces on the devices, where it is possible to identify the wireless interfaces called "w1p4s0" on both devices. Furthermore, it demonstrates the connectivity between said devices.

\begin{figure}
	\centering
	\includegraphics[width=\textwidth]{Chapters/Figures/tests/bmv2_phase_2/wireless_config_&_connectivity.PNG}
	\caption{Results from running the commands "ip a" and "ping" between the two devices on the network using the \gls{ip} address of the wireless interfaces.}
	\label{fig:exp2_phase2_wireless}
\end{figure}

As a result of this, two changes had to be made. The first of these amendments required the activation of the \gls{bmv2} with the wireless interface. The second required modification of the \gls{p4}/\gls{onos} applications and configuration to accommodate the distinctive characteristics of the wireless antennas. For any message to be successfully received by a wireless interface, that destination Media Access Control address must either be the address of the interface or the broadcast address. It was thus decided to pursue the latter course of action, which entailed changing the addresses of the messages sent via the wireless interface to the broadcast address. The original address is stored within the packet and inserted when required.

The test configuration is illustrated in Figure~\ref{fig:exp2_phase2_setup}, which was also used as the basis for the diagram seen in Figure~\ref{fig:exp2_phase2_diagram} and described in the following itemized list:

\begin{itemize}
	\item 192.168.137.1 - This \gls{ip} address identifies the author's personal laptop computer. It serves as the central coordinating hub for the entire network, facilitating the observation and monitoring of all network activity.
	\item 192.168.137.2 - This address represents a virtual machine that is hosted on the author's laptop via the Oracle VM VirtualBox virtualization platform. The virtual environment is configured with Debian 12 and is primarily intended for the deployment of \gls{onos}.
	\item 192.168.137.252 - This device corresponds to the version 2 of the devices. The device has a fresh installation of the Ubuntu Server operating system and is running \gls{bmv2} with grpc. Furthermore, the device is equipped with an antenna capable of communicating at frequencies up to 6 GHz and has been applied with the ath9k patch to the WiFi driver, thereby enabling communication using \gls{ieee} 802.11p.
	\item 192.168.137.253 - This device is virtually identical to the device 192.168.137.252.
	\item 10.10.10.1 - This \gls{ip} address identifies the virtual interface that has been added to the \gls{vrf} instance created in the 192.168.137.252 device.
	\item 10.10.10.2 - Identically to the last one, this \gls{ip} address identifies the virtual interface that has been added to the \gls{vrf} instance created in the 192.168.137.253 device.
\end{itemize}

\begin{figure}
	\centering
	\includegraphics[width=0.5\textwidth]{Chapters/Figures/tests/bmv2_phase_2/20241122_193213.jpg}
	\caption{Setup from the second phase of the experiments with \gls{bmv2}}
	\label{fig:exp2_phase2_setup}
\end{figure}

\begin{figure}
	\centering
	\includegraphics[width=0.7\textwidth]{Chapters/Figures/tests/bmv2_phase_2/setup_diagram.PNG}
	\caption{Diagram that represents the setup from the second phase of the experiments with \gls{bmv2}}
	\label{fig:exp2_phase2_diagram}
\end{figure}

When viewed through the lens of \gls{sdn}, the physical configuration can be conceptualized as illustrated in Figure~\ref{fig:exp2_phase2_sdn_diagram}.

\begin{figure}
	\centering
	\includegraphics[width=0.6\textwidth]{Chapters/Figures/tests/bmv2_phase_2/sdn_diagram.PNG}
	\caption{Diagram that depicts the \gls{sdn} perspective of the setup from the second phase of the experiments with \gls{bmv2}}
	\label{fig:exp2_phase2_sdn_diagram}
\end{figure}


Figure~\ref{fig:exp2_phase2_bmv2} demonstrates that the \gls{bmv2} program is operational in both devices, and that the programs have successfully connected to the wireless interfaces.

\begin{figure}
	\centering
	\includegraphics[width=\textwidth]{Chapters/Figures/tests/bmv2_phase_2/bmv2_running.PNG}
	\caption{Consoles running \gls{bmv2} on both devices in phase 2}
	\label{fig:exp2_phase2_bmv2}
\end{figure}


The list of devices and hosts known to the controller is shown in Figure~\ref{fig:exp2_phase2_onos}, and the flows that have been injected into the devices are presented in Figure~\ref{fig:exp2_phase2_onos_flows}. It is evident that both of these are virtually identical to those of the last phase.

\begin{figure}
	\centering
	\includegraphics[width=\textwidth]{Chapters/Figures/tests/bmv2_phase_2/onos_topology.PNG}
	\caption{\gls{onos} console with information relating to the devices and hosts of the network in phase 2}
	\label{fig:exp2_phase2_onos}
\end{figure}

\begin{figure}
	\centering
	\includegraphics[width=\textwidth]{Chapters/Figures/tests/bmv2_phase_2/onos_flows.PNG}
	\caption{\gls{onos} console with information relating to the flows of the network in phase 2}
	\label{fig:exp2_phase2_onos_flows}
\end{figure}

Figure~\ref{fig:exp2_phase2_pings} illustrates the connectivity between devices 192.168.137.252 and 192.168.137.253, manifesting through the VFRs with \glspl{ip} 10.10.10.1 and 10.10.10.2.

\begin{figure}
	\centering
	\includegraphics[width=\textwidth]{Chapters/Figures/tests/bmv2_phase_2/pings.PNG}
	\caption{Results from running the commands "ping" in the \gls{vrf} of the two devices on the network in phase 2}
	\label{fig:exp2_phase2_pings}
\end{figure}

% Analysis and Interpretation
\subsection{Analysis and Interpretation}

% Limitations
These experiments were constrained by a significant limitation, namely the availability of only two devices with \gls{bmv2} installed for use. As previously stated, the original device provided is not suitable for use in this scenario due to insufficient storage capacity to install \gls{bmv2}. Furthermore, financial constraints precluded the acquisition of another device of the newer version.
Obtaining more than two devices was crucial to effectively recreate more intricate testing scenarios that more closely resemble real-world contexts, such as the transmission of a message through a network of wireless devices. However, despite the inability to demonstrate this specific scenario due to a lack of resources, the observed functionality of the program, its code, and the logs from \gls{bmv2} and \gls{onos} provide strong indications that the code performs as intended in the aforementioned scenario.

In testing phase 1 of these tests, it was found that the default southbound \gls{api} of \gls{bmv2} is not P4Runtime, as previously assumed, but rather Thrift. The Thrift \gls{api}~\cite{noauthor_apache_nodate} is a component of the Apache Thrift framework, which originated at Facebook. The goal of Thrift is to facilitate cross-language communication and data serialization by providing a language-agnostic interface that can be implemented in a variety of programming languages. In this case, Thrift is used to turn an easy to read file with controller instructions to device specific instruction in the data plane.

This decision by the developers may initially seem incongruous, but it is, in fact, a logical consequence of the broader strategic context. \gls{bmv2} is not a production-grade switch. Rather, it is a software tool designed for educational and experimental purposes in conjunction with \gls{p4}. Thrift offers a more straightforward mechanism that streamlines the process of inserting rules on network devices. 
In consideration of this fact, it is clear that \gls{bmv2} was not developed with high performance as a primary objective. Accordingly, any practical development should seek to implement a faster solution in the event that performance issues arise.

The transposition of the final scenario to an \gls{sdn} network, illustrated in Figure~\ref{fig:exp2_phase2_sdn_diagram}, to an Vehicle \gls{its} sub-system has resulted in the diagram presented in Figure~\ref{fig:exp2_vehicle_subsystem}. From this diagram, a number of interesting observations and comparisons can be made. 
In this context, the \gls{sdn} switch can be compared to an \gls{its-s} gateway and an \gls{its-s} router. It is clear that any comparison should be viewed with caution, as the device under consideration does not align perfectly with any of the aforementioned categories. This comparison is intended to illustrate the device's functionality, rather than to provide a definitive categorization. It is inaccurate to categorize this device as TCP/IP to \gls{etsi} protocol translator, as it lacks the requisite functionality to convert commonly used Internet protocols into \gls{etsi} equivalents.
The \gls{sdn} controller could be integrated into the vehicle \gls{its} station as an \gls{its} application in a manner that would align with the standards and classifications set forth by the \gls{etsi}.

\begin{figure}
	\centering
	\includegraphics[width=0.6\textwidth]{Chapters/Figures/tests/bmv2_phase_2/its_vehicle_diagram.PNG}
	\caption{Diagram representing the \gls{sdn} network as an Vehicle \gls{its} sub-system}
	\label{fig:exp2_vehicle_subsystem}
\end{figure}

As anticipated, the \gls{bmv2} proved to be the optimal choice. This experiment was successful in enabling \gls{bmv2} to function with an \gls{onos} controller and the ath9k patch. The development of this test demonstrates that \gls{p4} is capable of functioning with \gls{ieee} 802.11p, which indicates that this technology can be utilized for vehicular deployment. The power and control provided by \gls{p4} affords users a vast array of potential applications. According to our tests it is possible to create a \gls{p4} based network that complies with \gls{etsi} norms, making it possible for this technology to be used in the real world.

